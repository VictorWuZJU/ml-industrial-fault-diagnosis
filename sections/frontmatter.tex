\begin{abstract}
Industrial cyber-physical systems increasingly rely on automated fault diagnosis to reduce downtime, improve safety, and enable predictive maintenance. Traditional model-based or rule-based approaches struggle with complex, nonlinear, and time-varying processes, motivating the use of data-driven machine learning methods. This paper proposes a unified mathematical formulation and a real-time prediction algorithm for machine-learning-based industrial fault diagnosis. First, we define a generic multi-sensor, multi-mode industrial process model and formalize the fault diagnosis problem as supervised classification over multivariate time-series windows. A compact representation of the data acquisition, feature extraction, and learning pipeline is developed, leading to an explicit mapping from sensor streams to fault labels and confidence scores. Second, we design a lightweight real-time prediction algorithm tailored for edge deployment, based on windowed streaming processing and a compressed neural classifier with bounded inference latency. The algorithm guarantees deterministic execution time per window and supports dynamic thresholding for early warning. Finally, we outline a benchmark protocol using publicly available rotating machinery and bearing datasets, as well as real plant data, to evaluate classification accuracy, false-alarm rate, and end-to-end latency. This framework provides a reusable blueprint for implementing industrial fault diagnosis systems that are mathematically transparent, implementable on resource-constrained hardware, and suitable for SCI-level empirical validation.
\end{abstract}

\noindent\textbf{Keywords:} Fault diagnosis, Industrial cyber-physical systems, Machine learning, Real-time prediction, Time-series classification, Edge computing.
